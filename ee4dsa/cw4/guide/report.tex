\documentclass[12pt,twoside]{report}

% Document dimensions and margins:
\usepackage[a4paper,width=170mm,top=25mm,bottom=25mm,bindingoffset=6mm]{geometry}

\usepackage[utf8]{inputenc}
\usepackage{graphicx}
\usepackage{float}
\usepackage{hyperref}
\usepackage{graphicx}
\usepackage{gensymb}
\usepackage[title]{appendix}
\usepackage[dotinlabels]{titletoc}
\usepackage[nottoc,numbib]{tocbibind}
\usepackage{mathtools}
\usepackage{enumitem}
\usepackage{longtable}

\restylefloat{figure}
\renewcommand{\thefootnote}{\arabic{footnote}}
\newcommand*\wrapletters[1]{\wr@pletters#1\@nil}
\def\wr@pletters#1#2\@nil{#1\allowbreak\if&#2&\else\wr@pletters#2\@nil\fi}
\setenumerate{itemsep=0pt}

\newcommand*\xor{\mathbin{\oplus}}

% Sort references by the order in which they are cited
\usepackage[style=authoryear,sorting=none]{biblatex}
\addbibresource{ref.bib}

% Fancy headers
\usepackage{fancyhdr}
\fancyhead{}
\fancyhead[RO,LE]{UCASM Assembler Reference}
\fancyfoot{}
\fancyfoot[LE,RO]{\thepage}
\fancyfoot[LO,CE]{Chapter \thechapter}
\fancyfoot[CO,RE]{Chris Cummins}
\pagestyle{fancy}


\begin{document}

\pagenumbering{roman}

%
% Title page. Original layout by Josh Cassidy. See:
%
%   https://www.sharelatex.com/blog/2013/08/09/thesis-series-pt5.html
%
\begin{titlepage}
  \begin{center}
    \vspace*{1cm}

    \Huge
    \textbf{UCASM Assembler Reference}

    \vspace{0.5cm}
    \LARGE
    EE4DSA Digital Systems Design

    \vspace{1.5cm}

    \textbf{Chris Cummins}

    \vfill

    MEng Electronic Engineering\\
    \& Computer Science

    \vspace{0.8cm}

    \Large
    Electronic Engineering\\
    School of Engineering and Applied Science\\
    Aston University\\
    March 2014

  \end{center}
\end{titlepage}


\tableofcontents

\newpage

\pagenumbering{arabic}

\chapter{Introduction}

The UCASM assembler is a Turing complete assembly language and macro
processor for the 32 bit uC architecture developed for coursework 3 of
Dr. Eberhard's Digital Systems Architecture module. The assembly
syntax and constructs are influenced heavily by the AVR Assembler
\footnote{See the AVR Assembler User Guide:
  \url{http://www.atmel.com/Images/doc1022.pdf}.}.

\section{Usage}

The core of the application is a multi-pass assembler written in
JavaScript, implemented as a module in the file
\texttt{lib/ee4dsa-assembler.js} which exposes a single function which
accepts three arguments: a string of data to assemble, a map of
options as key-value pairs, and a callback function for when the
assembly is completed.

There are two front-end clients which use this assembler function: a
standalone script written in Node, and an interactive web application
hosted at \url{http://chriscummins.cc/uc/assembler}. The standalone
assembler can be installed into the system path with the command
\texttt{npm install -g}, and accepts the following run time options:

\begin{verbatim}
Usage: ucasm -s <path> [options]

Options:
  --source, -s    Input source file                                   [required]
  --output, -o    Output RAM file path                   [default: "<source>.o"]
  --list, -l      Output listing file path               [default: "<source>.l"]
  --ram-size, -r  Set the size of the output RAM                 [default: 4096]
  --idt-size, -i  Set the size of the IDT                           [default: 8]
  --annotate, -a  Annotate the generated RAM file
\end{verbatim}

For example, to assemble the file \texttt{ram.asm} into an annotated
RAM file of 16k words, the client would be invoked as such:

\begin{verbatim}
$ ucasm --source ram.asm --ram-size 16384 --annotate
ram.asm: 240 words, 1.465% util (cseg: 92% dseg: 8%
\end{verbatim}

The file \texttt{ram.dat} now contains the generated annotated RAM
file:

\begin{verbatim}
$ head ram.o
08000000 -- 0 reti
020000DB -- 1 jmp 219
08000000 -- 2 reti
08000000 -- 3 reti
08000000 -- 4 reti
08000000 -- 5 reti
08000000 -- 6 reti
08000000 -- 7 reti
0600002D -- 8 call 45
020000B5 -- 9 jmp 181
\end{verbatim}

A list file is generated at \texttt{ram.l}, which contains a
diagnostic breakdown of the assembled program:

\begin{verbatim}
$ head ram.l
.SIZE = 16384

.IDT_SIZE = 8

.CSEG
        0 = reti
        1 = jmp 219
        2 = reti
        3 = reti
        4 = reti
\end{verbatim}

To perform the same assembly using the web client,

%% TODO:

\section{Nomenclature}

\subsection{Status Register (SREG)}

\begin{tabular}{ | r | l | }
  \hline
  \textbf{SREG}: & Status Register \\
  \textbf{I}: & Interrupts enabled flag \\
  \textbf{T}: & Test flag \\
  \textbf{C}: & Carry flag \\
  \hline
\end{tabular}

\subsection{Registers and Operands}

\begin{tabular}{ | r | l | }
  \hline
  \textbf{PC}: & Program Counter Register \\
  \textbf{STACK}: & Stack \\
  \textbf{SP}: & Stack Pointer Register \\
  \textbf{NULL}: & Null (zero) Register \\
  \textbf{$R_d$}: & Destination register in the Register File \\
  \textbf{$R_a$}: & Source register A in the Register File \\
  \textbf{$R_b$}: & Source register B in the Register File \\
  \textbf{$K$}: & Constant data \\
  \textbf{$k$}: & Constant address \\
  \textbf{$A$}: & I/O port address \\
  \textbf{$a$}: & I/O port AND mask \\
  \textbf{$x$}: & I/O port XOR mask \\
  \hline
\end{tabular}

\chapter{Assembly Syntax}

The assembler implements a line oriented parser which splits the input
source files first into separate lines and then into individual
tokens. A line may take one of the following five forms:

\begin{enumerate}
\item \texttt{label:}
\item \texttt{[label:] directive [operands] [Comment]}
\item \texttt{[label:] instruction [operands] [Comment]}
\item \texttt{Comment}
\item \texttt{Empty Line}
\end{enumerate}

Where a comment has the form:

\begin{verbatim}
; [Text]
\end{verbatim}

In the case where a line consists solely of a label, then the label
will refer to the contents of the line following it.

\section{Directives}

Directives control the internal state of the assembler, and do not
result in executable machine code being generated. Instead, they can
be used to set static addresses for program code, reserve memory for
variables and defined symbolic values.

{\scriptsize
\begin{longtable}{ | l | l | l | }
  \hline
  \textbf{Mnemonic} & \textbf{Operands} & \textbf{Description} \\
  \hline
\endfirsthead
  \hline
  \textbf{Mnemonic} & \textbf{Operands} & \textbf{Description} \\
  \hline

\endhead
  \hline
  \multicolumn{3}{r}{\emph{Continued on next page\ldots}}
\endfoot

\endlastfoot
 \hline
 .DEF & \textit{N, V} & Define a symbolic name \textit{N} which resolves to value \textit{V} \\
 .DEFP & \textit{N, V} & Define a symbolic name \textit{N} which resolves to value \textit{V} only if name \textit{N} has no existing definition \\
 .UNDEF & \textit{N} & Remove symbol \textit{N} and its corresponding value from the symbol table \\
 .CSEG & & Code Segment \\
 .DSEG & & Data Segment \\
 .ORG & \textit{k} & Set program origin to address \textit{k} \\
 .ISR & \textit{n, k} & Set ISR handler \textit{n} to address \textit{k} \\
 .EXIT & & Stop assembling \\
 .INCLUDE & \textit{f} & Read source from file \textit{f} (standalone assembler only) \\
 .BYTE & \textit{n} & Reserve \textit{n} bytes of program memory \\
 .WORD & \textit{n} & Reserve \textit{n} words of program memory \\
 \hline
\end{longtable}}

\section{Instruction Set}

{\scriptsize
\begin{longtable}{ | l | l | l | c | l | l | }
  \hline
  \textbf{Mnemonic} & \textbf{Operands} & \textbf{Description} & \textbf{Operation} & \textbf{Flags} & \textbf{\#Clocks} \\
  \hline
\endfirsthead
  \hline
  \textbf{Mnemonic} & \textbf{Operands} & \textbf{Description} & \textbf{Operation} & \textbf{Flags} & \textbf{\#Clocks} \\
  \hline

\endhead
  \hline
  \multicolumn{6}{r}{\emph{Continued on next page\ldots}}
\endfoot

\endlastfoot
NOP & & No operation & & None & 1 \\
HALT & & Halt CPU & $PC \leftarrow PC$ & None & - \\
JMP & $k$ & Jump & $PC \leftarrow k$ & None & 1 \\
RJMP & $k$ & Relative Jump & $PC \leftarrow PC + k$  & None & 1 \\
BRTS & $k$ & Branch if T flag set & if (T = 1) then $PC \leftarrow k$  & None & 1 \\
RBRTS & $k$ & Relative Br. if T flag set & if (T = 1) then $PC \leftarrow PC + k$  & None & 1 \\
BREQ & $R_a, R_b, k$ & Br. if equal & if ($R_a = R_b$) then $PC \leftarrow k$ & T & 4* \\
BREQI & $R_a, K, k$ & Br. if equal immediate & if ($R_a = K$) then $PC \leftarrow k$ & T & 8* \\
BRNE & $R_a, R_b, k$ & Br. if not equal & if ($R_a \neq R_b$) then $PC \leftarrow k$ & T & 4* \\
BRNEI & $R_a, K, k$ & Br. if not equal immediate & if ($R_a \neq K$) then $PC \leftarrow k$ & T & 8* \\
BRLT & $R_a, R_b, k$ & Br. if less than & if ($R_a < R_b$) then $PC \leftarrow k$ & T & 4* \\
BRLTE & $R_a, R_b, k$ & Br. if less than or equal & if ($R_a \le R_b$) then $PC \leftarrow k$ & T & 4* \\
BRGT & $R_a, R_b, k$ & Br. if greater than & if ($R_a > R_b$) then $PC \leftarrow k$ & T & 4* \\
BRGTE & $R_a, R_b, k$ & Br. if greater than or equal & if ($R_a \ge R_b$) then $PC \leftarrow k$ & T & 4* \\
SETO & $A, a, x$ & Set output port & $A \leftarrow A . a \xor x$  & None & 1 \\
TSTI & $A, a, x$ & Read input port & $T \leftarrow A . a \xor x$  & T & 1 \\
CALL & $k$ & Call subroutine & $PC \leftarrow k$ & None & 1 \\
RCALL & $k$ & Relative call subroutine & $PC \leftarrow PC + k$ & None & 1 \\
RET & & Return from subroutine & $PC \leftarrow STACK$ & None & 2 \\
RETI & & Return from interrupt & $PC \leftarrow STACK$ & I & 2 \\
SEI & & Global enable interrupts & $I \leftarrow 1$ & I & 1 \\
CLI & & Global disable interrupts & $I \leftarrow 0$ & I & 1 \\
LD & $R_d, k$ & Load & $R_d \leftarrow RAM[k]$ & None & 2 \\
LDI & $R_d, K$ & Load immediate & $R_d \leftarrow K$ & NONE & 4* \\
LDIH & $R_d, K$ & Load immediate high & $R_d(31:16) \leftarrow K$ & NONE & 2 \\
LDIL & $R_d, K$ & Load immediate low & $R_d(15:0) \leftarrow K$ & NONE & 2 \\
LDR & $R_d, R_a$ & Load indirect & $R_d \leftarrow RAM[R_a]$ & NONE & 3 \\
LDD & $R_d, R_a, R_b$ & Load indirect offset & $R_d \leftarrow RAM[R_a + R_b]$ & None & 3 \\
LDDI & $R_d, R_a, k$ & Load indirect offset immediate & $R_d \leftarrow RAM[R_a + k]$ & NONE & 7* \\
LDIO & $R_d, A$ & IO to Register & $R_d \leftarrow A$ & None & 1 \\
ST & $k, R_a$ & Write & $RAM[k] \leftarrow R_a$ & None & 2 \\
STI & $k, K$ & Write immediate & $RAM[k] \leftarrow K$ & None & 6* \\
STR & $R_d, R_a$ & Write indirect & $RAM[R_d] \leftarrow R_a$ & None & 3 \\
STRI & $R_d, K$ & Write indirect immediate & $RAM[R_d] \leftarrow K$ & None & 7* \\
STD & $R_d, R_b, R_a$ & Write indirect offset & $RAM[R_d + R_b] \leftarrow R_a$ & None & 3 \\
STDI & $R_d, k, R_a$ & Write indirect offset immediate & $RAM[R_d + k] \leftarrow R_a$ & None & 7* \\
STIO & $A, R_a$ & Register to IO & $A \leftarrow R_a$ & None & 2 \\
PSHR & $R_a$ & Push register on stack & $STACK \leftarrow R_a$ & None & 2 \\
PSHI & $K$ & Push immediate on stack & $STACK \leftarrow K$ & None & 6* \\
POPR & $R_d$ & Pop register from stack & $R_d \leftarrow STACK$ & None & 2 \\
AND & $R_d, R_a, R_b$ & Logical AND & $R_d \leftarrow R_a . R_b $ & None & 3 \\
ANDI & $R_d, R_a, K$ & Logical AND immediate & $R_d \leftarrow R_a . K$ & None & 7* \\
OR & $R_d, R_a, R_b$ & Logical OR & $R_d \leftarrow R_a + R_b $ & None & 3 \\
ORI & $R_d, R_a, K$ & Logical OR immediate & $R_d \leftarrow R_a + K $ & None & 7* \\
XOR & $R_d, R_a, R_b$ & Logical XOR & $R_d \leftarrow R_a \xor R_b $ & None & 3 \\
XORI & $R_d, R_a, K$ & Logical XOR immediate & $R_d \leftarrow R_a \xor K $ & None & 7* \\
LSR & $R_d, R_a, R_b$ & Logical Shift Right & $R_d(n) \leftarrow R_d(n + R_b) $ & None & $R_b$* \\
LSRI & $R_d, R_a, K$ & Logical Shift Right Immediate & $R_d(n) \leftarrow R_d(n + K) $ & None &  \\
LSL & $R_d, R_a, R_b$ & Logical Shift Left & $R_d(n) \leftarrow R_d(n - R_b) $ & None & $R_b$* \\
LSLI & $R_d, R_a, K$ & Logical Shift Left Immediate & $R_d(n) \leftarrow R_d(n - K) $ & None & 3 \\
EQU & $R_a, R_b$ & Equal & if ($R_a = R_b$) then $T \leftarrow 1$ & T & 3 \\
EQI & $R_a, K$ & Equal Immediate & if ($R_a = K$) then $T \leftarrow 1$ & T & 7* \\
NEQ & $R_a, R_b$ & Not Equal & if ($R_a \neq R_b$) then $T \leftarrow 1$ & T & 3 \\
NEQI & $R_a, K$ & Not Equal Immediate & if ($R_a \neq K$) then $T \leftarrow 1$ & T & 7* \\
LT & $R_a, R_b$ & Less Than & if ($R_a < R_b$) then $T \leftarrow 1$ & T & 3 \\
LTI & $R_a, K$ & L.T. Immediate & if ($R_a < K$) then $T \leftarrow 1$ & T & 7* \\
LTS & $R_a, R_b$ & L.T., Signed & if ($R_a < R_b$) then $T \leftarrow 1$ & T & 3 \\
LTSI & $R_a, K$ & L.T. Immediate, Signed & if ($R_a < K$) then $T \leftarrow 1$ & T & 7* \\
LTE & $R_a, R_b$ & L.T. or Equal & if ($R_a \le R_b$) then $T \leftarrow 1$ & T & 3 \\
LTEI & $R_a, K$ & L.T. or Equal Immediate & if ($R_a le K$) then $T \leftarrow 1$ & T & 7* \\
LTES & $R_a, R_b$ & L.T. or Equal, Signed & if ($R_a \le R_b$) then $T \leftarrow 1$ & T & 3 \\
LTESI & $R_a, K$ & L.T. or Equal Immediate, Signed & if ($R_a \le K$) then $T \leftarrow 1$ & T & 7* \\
GT & $R_a, R_b$ & Greater Than & if ($R_a > R_b$) then $T \leftarrow 1$ & T & 3 \\
GTI & $R_a, K$ & G.T. Immediate & if ($R_a > K$) then $T \leftarrow 1$ & T & 7* \\
GTS & $R_a, R_b$ & G.T., Signed & if ($R_a > R_b$) then $T \leftarrow 1$ & T & 3 \\
GTSI & $R_a, K$ & G.T. Immediate, Signed & if ($R_a > K$) then $T \leftarrow 1$ & T & 7* \\
GTE & $R_a, R_b$ & G.T. or Equal & if ($R_a \ge R_b$) then $T \leftarrow 1$ & T & 3 \\
GTEI & $R_a, K$ & G.T. or Equal Immediate & if ($R_a \ge K$) then $T \leftarrow 1$ & T & 7* \\
GTES & $R_a, R_b$ & G.T. or Equal, Signed & if ($R_a \ge R_b$) then $T \leftarrow 1$ & T & 3 \\
GTESI & $R_a, K$ & G.T. or Equal Immediate, Signed & if ($R_a \ge K$) then $T \leftarrow 1$ & T & 7* \\
EQZ & $R_a$ & Equal Zero & if ($R_a = 0$) then $T \leftarrow 1$ & T & 3 \\
NEZ & $R_a$ & Not Equal Zero & if ($R_a \neq 0$) then $T \leftarrow 1$ & T & 3 \\
MOV & $R_d, R_a$ & Copy Register contents & $R_d \leftarrow R_a$ & None & 3 \\
CLR & $R_d$ & Clear Register & $R_d \leftarrow 0$ & None & 3 \\
NEG & $R_d$ & Two's Complement & $R_d \leftarrow 0 - R_d$ & None & 3 \\
INC & $R_d$ & Increment & $R_d \leftarrow R_d + 1$ & None & 3 \\
INCS & $R_d$ & Increment, Signed & $R_d \leftarrow R_d + 1$ & None & 3 \\
DEC & $R_d$ & Decrement & $R_d \leftarrow R_d - 1$ & None & 3 \\
ADD & $R_d, R_a, R_b$ & Add & $R_d \leftarrow R_a + R_b$ & C & 3 \\
ADDI & $R_d, R_a, K$ & Add Immediate & $R_d \leftarrow R_a + K$ & C & 7* \\
ADDS & $R_d, R_a, R_b$ & Add, Signed & $R_d \leftarrow R_a + R_b$ & C & 3 \\
ADDSI & $R_d, R_a, K$ & Add Immediate, Signed & $R_d \leftarrow R_a + K$ & C & 7* \\
SUB & $R_d, R_a, R_b$ & Subtract & $R_d \leftarrow R_a - R_b$ & C & 3 \\
SUBI & $R_d, R_a, K$ & Subtract Immediate & $R_d \leftarrow R_a - K$ & C & 7* \\
SUBS & $R_d, R_a, R_b$ & Subtract, Signed & $R_d \leftarrow R_a - R_b$ & C & 3 \\
SUBSI & $R_d, R_a, K$ & Subtract Immediate, Signed & $R_d \leftarrow R_a - K$ & C & 7* \\
 \hline
\end{longtable}}

Instructions which have an asterisk (*) mark in the \#Clocks column
indicate macro instructions, which are non-native software
instructions which expand to two or more hardware instructions.

\section{Symbols}

The following symbols are defined at initialisation time by the
assembler. Symbols marked as Dynamic in the type column have a value
which is determined for each time it is used, as opposed to the other
symbols which have a static value associated with them which does not
change. It is possible although not recommended to override any of
these symbols using the \texttt{.DEF} directive.

{\scriptsize
\begin{longtable}{ | l | l | l | }
  \hline
  \textbf{Symbol} & \textbf{Type} & \textbf{Description} \\
  \hline
\endfirsthead
  \hline
  \textbf{Symbol} & \textbf{Type} & \textbf{Description} \\
  \hline

\endhead
  \hline
  \multicolumn{3}{r}{\emph{Continued on next page\ldots}}
\endfoot

\endlastfoot
 \hline
 RAM\_SIZE & Integer & The size of the assembled RAM file, in words \\
 IDT\_SIZE & Integer & The size of the interrupt service routine table, in words \\
 IDT\_START & Integer (default: 0) & The start address of the interrupt service routine table \\
 PROG\_START & Integer (default: IDT\_SIZE) & The start address of the executable program code \\
 \_\_R & Register (default: r4) & The register used by the assembler to implement macro instructions \\
 ACTIVE\_ADDRESS & Integer, Dynamic & The currently active RAM address \\
 ACTIVE\_SEGMENT & Integer, Dynamic & The currently active segment type \\
 CSEG\_SIZE & Integer, Dynamic & The current total size of the code segments \\
 DSEG\_SIZE & Integer, Dynamic & The current total size of the data segments \\
 \hline
\end{longtable}}

One possible legitimate purpose for overriding the built in symbols
would be to declare a different register to use for macro
instructions:

\begin{verbatim}
        ldi r16, 0xDEADBEEF             ; Macro instruction using default __r
        .DEF __r r255                   ; Override default __r address
        ldi r17, 0xDEADBEEF             ; Macro instruction using r255 as
\end{verbatim}

Symbols are resolved recursively during tokenization, and so can be
combined to provide complex evaluations:

\begin{verbatim}
        .def TABLE_ADDR     0xff
        .def NO_OF_ELEMENTS 10
        .def ELEMENT_SIZE   4
        .def TABLE_SIZE     ELEMENT_SIZE * NO_OF_ELEMETNS

        sti TABLE_ADDR,                1 ; foobar[0] = 1
        sti TABLE_ADDR + ELEMENT_SIZE, 2 ; foobar[1] = 2

        .dseg
        .org TABLE_ADDR
foobar: .word TABLE_SIZE
\end{verbatim}

\chapter{Assembler Components}

The assembler performs multiple passes of the input source files,
generating an internal representation of the described program, which
is then written out into different forms. Overall, there are four
distinct high level stages to the assembly process:

\begin{enumerate}
\item \textbf{Pre-process} - Recursively expand multi-instruction macros.
\item \textbf{First pass} - Perform tokenization of every line and
  interpret instructions, creating internal lookup tables for memory
  addresses, labels, and symbols.
\item \textbf{Second pass} - resolve memory and label names using
  populated lookup tables from first pass, creating an intermediate
  representation of the program.
\item \textbf{Assembly} - Convert intermediate representation of the
  program into machine instructions, and generate a list file.
\end{enumerate}

\section{Pre-processor}

The native $\mu C$ design implements a load/store architecture with 43
unique instructions. Of which, only 2 operations accept immediate
operands, with the rest requiring register operands. While this in no
way limits the computational power of the processor, it can make
programming the processor quite tedious, as any instruction involving
immediate data must first be loaded into registers in prior
instructions before use. In order to produce a more assembly programs,
a number of instructions which operate on immediate values have been
implemented as macros which are internally substituted for a load
instruction to a temporary register (defined in the symbol
\texttt{\_\_r}), followed by a native instruction which operates on
this register. For example, the instruction:

\begin{verbatim}
        stri r5, 0xDEADBEEF             ; Store constant at address r5.
\end{verbatim}

is expanded at the pre-processing stage to equate to:

\begin{verbatim}
        ldih __r, 0xDEAD                ; Store 16 high bits to temp register.
        ldil __r, 0xBEEF                ; Store 16 low bits to temp register.
        str r5, __r                     ; Store __r at address r5.
\end{verbatim}

This allows for a more flexible instruction set, with the
pre-processing stage recursively expanding any macros in the input
source code by using a lookup table to expand macro instructions into
native instructions. The downside to this is that the temporary
register \texttt{\_\_r} used to implement these instructions should
not be used by the user as this could lead to undefined behaviour.

\section{First pass unit}

Once the pre-processor has expanded all macros, the assembler parses
the pre-processed text line by line, sequentially tokenizing each one
before proceeding to the next. The tokenization process involves eight
distinct steps, illustrated here by using an example input line,
containing a Load Immediate instruction:

\begin{verbatim}
        ldil SP, 0xFF00 >> 16           ; Load 255 into register 90.
\end{verbatim}

\begin{enumerate}
\item Strip whitespace and comment: \texttt{LDIL SP, 0xFF00 >> 16}
\item Convert line text to lowercase: \texttt{ldil SP, 0xff00 >> 16}
\item Split line text into individual words.
\item Strip trailing commas from words: \texttt{ldil SP 0xff00 >> 16}
\item Recursively resolve symbols for each word: \texttt{ldil r2 0xff00 >> 16}
\item Internally convert numbers into signed decimal integers: \texttt{ldil 2 65280 >> 16}
\item Resolve numerical modifiers (e.g. bitwise complement).
\item Recursively resolve numerical and bitwise expressions: \texttt{ldil 2 255}
\end{enumerate}

Once the line has been tokenized, the individual tokens are iterated
over and decoded into an instruction or directive command as
required. The assembler maintains an internal state machine tracking
the current RAM address, current segment type, and a set of tables
which are used to perform lookups for symbolic references, memory
addresses and label addresses.

\section{Second pass unit}

After the first pass, the lookup tables are fully populated, and the
tokens of the program are iterated over once more, this time replacing
any occurrences of labels or variable names with the addresses found
within the lookup tables. For example, consider the following example
code:

\begin{verbatim}
_main:
        jmp _start              ; Jump ahead
        .dseg
foo:    .byte 3                 ; Allocate 3 bytes with label 'foo'
bar:    .word 5                 ; Allocate 5 words with label 'bar'
        .cseg
        .org 0xF                ; Set an exact address for the following code
_start:
        halt                    ; Stop the processor
\end{verbatim}

After assembly, the list file contains a report which shows, amongst
other things, the contents of the internal memory and label tables:

\begin{verbatim}
.MEMORY
        foo = 1
        bar = 2
.LABELS
        _main = 0
        _start = 15
\end{verbatim}

Meaning that the two memory variables \texttt{foo} and \texttt{bar}
have been allocated addresses 1 and 2 respectively, and the two labels
have the addresses 0 and 15. It is important to note that the code and
data segments are not allocated contiguously, and that code sections
and data segments will be allocated ascending memory addresses
determined by their order within the source file.

\section{Machine Code Generator}

Once the internal representation of the program has been completed and
all memory addresses and symbols have been resolved, the output file
is generated, which translates the processed instructions into machine
code, using internal reference tables to encode the opcode and
operands. The example assembly code above generates the following
output file when assembled with the \texttt{--annotated} flag set:

\begin{verbatim}
0200000F -- 0 jmp 15
00000000 -- 1 DATA: foo
00000000 -- 2 DATA: bar[0]
00000000 -- 3 DATA: bar[1]
00000000 -- 4 DATA: bar[2]
00000000 -- 5 DATA: bar[3]
00000000 -- 6 DATA: bar[4]
...
01000000 -- 15 halt
\end{verbatim}

It is at this late stage in the assembly process that type checking of
operands are performed, with detection of malformed instructions
occurring and throwing fatal errors where found. The combination of
this late type checking and the early decimal conversion that occurs
within the tokenization phase makes for a very expressive assembly
syntax, at the expense of strict static typing. Some examples of this
include:

\begin{verbatim}
_main:
        rjmp  _start - CURRENT_ADDRESS  ; Relative address arithmentic

_start:
        ldil PC, CURRENT_ADDRESS + 1    ; Load next instruction into PC
        jmp 3 + _start                  ; Address arithmetic
        ldi 15 * 2, 60 / 2              ; Load 30 into r30
        ldi r10, foobar                 ; Load the address of 'foobar'
        ld r11, foobar                  ; Load the value at 'foobar'

        .dseg
foobar: .word 10 * 3                    ; Allocate 30 words to 'foobar'
\end{verbatim}

\chapter{Programming with UCASM}

\section{Working with Memory}

Memory may be reserved and assigned to a symbolic label by using the
data segment directive. Memory is allocated at the moment of
declaration, so the data segment does not form a contiguous block, but
is intermingled with the code segment. For example, the source code:

\begin{verbatim}
        .dseg
        .org 0xf
foo:    .byte 1
        .cseg
        ld r10, bar + 2
        rjmp 4
        .dseg
bar:    .word 3
        .cseg
        halt
\end{verbatim}

Will generate the following output machine code:

\begin{verbatim}
00000000 -- 15 DATA: foo
0B0A0014 -- 16 ld r10 20
02000015 -- 17 rjmp 4
00000000 -- 18 DATA: bar[0]
00000000 -- 19 DATA: bar[1]
00000000 -- 20 DATA: bar[2]
01000000 -- 21 halt
\end{verbatim}

Note that reserved memory is not initialised.

\section{Interrupt Handlers}

Interrupt handlers are assigned using the \texttt{ISR} directive. At
assembly time, an empty interrupt service handler table is generated,
and any registered ISR are added. For example, the source code:

\begin{verbatim}
        .isr 0 isr0             ; Register the ISR 0 handler using label
        .isr 1 0xf0             ; Register the ISR 1 handler using address
                                ; Note that ISR 2 and above are not declared
        .cseg
        halt

isr0:                           ; ISR 0
        pshr r16
        ldio r16, 0
        stio 0, r16
        popr r16

        .org 0xf0               ; ISR 1
        seto 0, 0, 0
        reti
\end{verbatim}

Generates the following machine code when assembled with IDT size of 3
words set:

\begin{verbatim}
02000004 -- 0 jmp 4
020000F0 -- 1 jmp 240
08000000 -- 2 reti
01000000 -- 3 halt
0F100000 -- 4 pshr r16
12100000 -- 5 ldio r16 0
11001000 -- 6 stio 0 r16
10100000 -- 7 popr r16
...
04000000 -- 240 seto 0 0 0
08000000 -- 241 reti
\end{verbatim}

\section{stdlib.asm}

\subsection{Symbols}

{\scriptsize
\begin{longtable}{ | l | l | l | }
  \hline
  \textbf{Symbol} & \textbf{Type} & \textbf{Description} \\
  \hline
\endfirsthead
  \hline
  \textbf{Symbol} & \textbf{Type} & \textbf{Description} \\
  \hline

\endhead
  \hline
  \multicolumn{3}{r}{\emph{Continued on next page\ldots}}
\endfoot

\endlastfoot
 \hline
 \_STDLIB\_USER\_ENTRY\_POINT & Label* & User code entry point \\
 \_STDLIB\_REG\_BASE & Integer* & Base address for internal register usage \\
 \$r to \$rf & Integer* & 16 Register File addresses for internal use \\
 STDLIB\_REG\_MIN & Integer & Lower Register File address for internal use \\
 STDLIB\_REG\_MAX & Integer & Upper Register File address for internal use \\
 NULL & Register & Null Register \\
 PC & Register & Program Counter Register \\
 SP & Register & Stack Pointer Register \\
 SREG & Register & Status Register \\
 ISR\_TIMER & ISR & Timer interrupt number \\
 ISR\_SSD & ISR & Seven Segment Display interrupt number \\
 SREG\_I & Integer & SREG Interrupts enabled flag index \\
 SREG\_T & Integer & SREG Test flag index \\
 SREG\_C & Integer & SREG Carry flag index \\
 SWITCHES & Port & Switches input port address \\
 BUTTONS & Port & Buttons input port address \\
 LEDS & Port & LEDs output port address \\
 SSD\_AN & Port & Seven Segment Display Anodes output port address \\
 SSD\_KA & Port & Seven Segment Display Cathodes output port address \\
 BTND & Integer & Down button bit mask position \\
 BTNC & Integer & Centre button bit mask position \\
 BTNL & Integer & Left button bit mask position \\
 BTNR & Integer & Right button bit mask position \\
 SSD\_CHAR\_0 to SSD\_CHAR\_9 & Byte & Seven Segment Display cathode masks for digits 0 to 9 \\
 SSD\_PEROD & Byte & Seven Segment Display cathode mask for period \\
 SSD\_OFF & Byte & Seven Segment Display cathode mask with all segments disabled \\
 \hline
\end{longtable}}

\subsection{Subroutines}

{\scriptsize
\begin{longtable}{ | l | l | l | l | }
  \hline
  \textbf{Label} & \textbf{Parameters} & \textbf{Return} & \textbf{Description} \\
  \hline
\endfirsthead
  \hline
  \textbf{Label} & \textbf{Parameters} & \textbf{Return} & \textbf{Description} \\
  \hline

\endhead
  \hline
  \multicolumn{4}{r}{\emph{Continued on next page\ldots}}
\endfoot

\endlastfoot
 \hline
 btnu\_press & & & Wait for the user to press and release up button \\
 btnd\_press & & & Wait for the user to press and release down button \\
 btnc\_press & & & Wait for the user to press and release center button \\
 btnl\_press & & & Wait for the user to press and release left button \\
 btnr\_press & & & Wait for the user to press and release right button \\
 bcd2ssd & \textit{B} & \textit{S} & Converts BCD digit \textit{B} into SSD digit \textit{S} \\
 bcd2ssd\_p & \textit{B} & \textit{S.} & Converts BCD digit \textit{B} into SSD digit \textit{S.} with period segment lit \\
 bcd2ssd\_tm & \textit{k, U} & & Convert unsigned integer \textit{U} into 4-digit SSD cathode mask table at address \textit{k} \\
 mult & \textit{A, B} & \textit{R} & Multiply integers \textit{A} and \textit{B} and return product \textit{R} \\
 div & \textit{A, B} & \textit{R, r} & Divide numerator \textit{A} by denominator \textit{B} and return result \textit{R} and remainder \textit{r} \\
 sortu & \textit{n, k} & & In-place unsigned integer array sort of array size \textit{n} at address \textit{k} \\
 sorts & \textit{n, k} & & In-place signed integer array sort of array size \textit{n} at address \textit{k} \\
 \hline
\end{longtable}}

\end{document}
